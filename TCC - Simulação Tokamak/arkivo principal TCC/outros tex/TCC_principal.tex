\documentclass[12pt,oneside,a4paper]{abntex2}                   
\usepackage{cmap}                               % Mapear caracteres especiais no PDF
\usepackage{lmodern}                    % Usa a fonte Latin Modern                      
\usepackage[usenames,dvipsnames]{pstricks}
\usepackage{graphicx}
\usepackage{epstopdf}
\usepackage{enumerate}
\usepackage{amsthm}
%\usepackage{amsmath}
\usepackage{multicol}
\usepackage{mathptmx}
\usepackage{esvect}
\usepackage{amssymb,indentfirst}  
\usepackage[centertags]{amsmath} 
\usepackage[T1]{fontenc}        
\usepackage[utf8]{inputenc}          
\usepackage{makeidx}            % Cria o indice
\usepackage{hyperref}                   % Controla a formação do índice
\usepackage{lastpage}                   % Usado pela Ficha catalográfica
\usepackage{indentfirst}              
\usepackage{nomencl}           
\usepackage{color}                             
\usepackage{graphicx}             
\usepackage{pdfpages}
%%% pacotes q eu adicionei


%%% fim dos pacotes adicionados por mim


\usepackage[normalem]{ulem}     % Sublinhados
% ---
 \definecolor{lightblue}{rgb}{0.68,0.85,0.9}
 \definecolor{indianred}{rgb}{0.8,0.36,0.36}

\usepackage[brazilian,hyperpageref]{backref}     % Paginas com as citações na bibl
\usepackage[alf]{abntex2cite}   % Citações padrão ABNT
\usepackage[usenames,dvipsnames]{xcolor}
\definecolor{cinza}{rgb}{0.95,0.95,0.95}
\newcommand{\quadro}[1]{\fcolorbox{white}{cinza}{\begin{minipage}[t]{0.9\textwidth} #1 \end{minipage} }}   

\theoremstyle{definition}  %Tira o itálico dos newtheorem
\newtheorem{definition}{Definição}[section]
\newtheorem{theorem}{Teorema}[section]
\newtheorem{obs}{Observação}[section] 
\newtheorem{com}{Comentário}[section] 
\newcommand{\PP}{\mathbb P}
\newcommand{\TT}{\mathbb T}
\newcommand{\RR}{\mathbb R}
\newcommand{\ZZ}{\mathbb Z}
\newcommand{\NN}{\mathbb N}
\newcommand{\CC}{\mathbb C}
\newcommand{\VV}{\mathbb V}
\newcommand{\sen}{\mathrm{sen}}

\renewcommand{\backrefpagesname}{Citado na(s) página(s):~}
% Texto padrão antes do número das páginas
\renewcommand{\backref}[1]{}
% Define os textos da citação
\renewcommand*{\backrefalt}[4]{
        \ifcase #1 %
                Nenhuma citação no texto.%
        \or
                Citado na página #2.%
        \else
                Citado #1 vezes nas páginas #2.
        \fi}%

\titulo{{Modelo de 2 Fluidos para o Breackdown no Tocamak Nova-Furg}}
\autor{Kévi Pegoraro}
\local{Rio Grande, Rio Grande do Sul, Brasil}
\data{Mês da Defesa, Ano}
\orientador{{Dro. Magno P. Coralles}}
\coorientador{{Dr. se houver}}
\instituicao{%
  Universidade Federal do Rio Grande - FURG
  \par
  Instituto de Matemática, Estatística e Física - IMEF
  \par
  Curso de Matemática Aplicada Bacharelado}
\tipotrabalho{Trabalho de Conclusão de Curso}
\preambulo{Trabalho de Conclusão de Curso, Matemática Aplicada Bacharelado, submetido por XXXX junto ao Instituto de Matemática, Estatística e Física da Universidade Federal do Rio Grande.}
\definecolor{blue}{RGB}{41,5,195}

\setlength{\parindent}{1.3cm}

\setlength{\parskip}{0.2cm}  % tente também 
\makeindex

\makenomenclature

\begin{document}

\imprimircapa

\imprimirfolhaderosto*

\newpage

\begin{dedicatoria}
   \vspace*{\fill}
   \centering
   \noindent
   \textit{Este trabalho é dedicado Liberdade que não temos.}
   
   \vspace{.3cm}
% \begin{citacao}   Se és capaz de aceitar teus alunos como são, com suas diferentes realidades humanas, sociais e culturais; se os levas a superar as dificuldades, 
% limitações ou fracassos, sem humilhações, sem inúteis frustrações; se os levas a refletir mais do que decorar; se te emocionas com a visão de tantas criaturas que de 
% ti dependem para desabrochar em consciência, criatividade, liberdade e responsabilidade, então podes dizer: sou mestre. 
% \end{citacao} (Rui Barbosa) \vspace*{\fill}
\end{dedicatoria}
% ---
\newpage

\begin{agradecimentos}
Agradeço ao professor Magno Pinto Collares pela grande ajuda na oreientação deste trabalho e ao professor Gustavo Paganini Canal - USP pela  contribuição com material de estudo, resposta a inúmeras duvidas e ajuda com deselvolviemnto númerico.
\end{agradecimentos}

\begin{resumo}
XXXXXX. 
\vspace{\onelineskip}
\noindent \textbf{Palavras-chaves}: XX, XX.
\newpage 
  
%
%    
\end{resumo}
% resumo em inglês
\begin{resumo}[Abstract]
This XXXXXX. 
\vspace{\onelineskip}
 
 \noindent 
 \textbf{Key-words}: XX, XX.
\end{resumo}

\pdfbookmark[0]{\listfigurename}{lof}
\listoffigures*
\cleardoublepage

 \pdfbookmark[0]{\listtablename}{lot}
 \listoftables*
 \cleardoublepage

\pdfbookmark[0]{\contentsname}{toc}
\tableofcontents*
\cleardoublepage
% ---
% ----------------------------------------------------------
% ELEMENTOS TEXTUAIS
% ----------------------------------------------------------
% É possível usar \textual ou \mainmatter, que é a macro padrão do memoir.  
\mainmatter
\chapter*[Introdução]{Introdução}
\addcontentsline{toc}{chapter}{Introdução}
Entender o breakdown do plasma é essencial para melhorar o desempenho dos tokamaks e é um passo importante para chegarmos a fonte de energia ampla e não poluente que a fusão pode se tornar.
O objetivo do presente trabalho é, por meio de uma simulação númerica, estudar  a fase de brakdown do plasma dentro de um tokamak, do inicio da descarga primária até a estabilização dentro to tokamak NOVA-Furg. Espero neste trabalho obter dados de simulação querentes com os medidos nos disparos do tokamak, para poder coompreender melhor oque esta acontecendo durante o breakdownn, tendo assim uma maneira de prever os resultados de cada experiemento na maquina.
\section{Como Funciona um Tokamak}
Tokamak é um reator experimental de fusão nuclear. Serve para estudar plasmas de alta tempertura que são mantidos confinados por campos magnéticos intensos. O objetivo final da pesquisa nesta área é viabilizar, no futuro, a construção de reatores nucleares de fusão, onde núcleos de deutério e trítio possam se unir, liberando uma grande quantidade de energia que servirá para aquecer água, gerar vapor e assim mover uma turbina, acoplada a um gerador elétrico. A pesquisa em tokamaks, portanto, está ligada à procura de fontes alternativas de energia para a produção de eletricidade. 
Basicamente, o Tokamak é um potente eletroímã que produz um campo magnético toroidal para o confinamento de plasma (o quarto estado físico da matéria, que compõe as estrelas) de isótopos pesados de hidrogênio (deutério e trítio especificamente). Em seu interior ocorre uma reação de fusão nuclear cujo objetivo é criar plasma que deve ser contido em um espaço limitado, de forma a não tocar nas paredes internas do reator, tanto para não danificá-lo, quanto para não dissipar a energia do combustível via condução térmica. O plasma é então contido pelo intenso campo magnético gerado pelo Tokamak.
O isolamento magnético permite que se alcancem altas temperaturas, impedindo o combustível da reação, os isótopos de hidrogênio, de desgastar ou sobreaquecer o reator. O campo magnético tem geometria toroidal (em forma de pneu). Este método de contenção do plasma, é conhecido como confinamento magnético. Existe ainda, outra forma de confinamento do plasma que é o confinamento inercial. Nesta, um laser de alta potência bombardeia o combustível do reator. Isto causa a "implosão" do combustível e o início de uma reação em cadeia, que tem como consequência o início do processo de fusão nuclear. Na natureza há uma terceira forma, o confinamento gravitacional, este impraticável na Terra. O confinamento gravitacional é a forma como as estrelas contêm o plasma. O Sol, assim como todas as estrelas, é na verdade um reator natural de fusão nuclear. O Tokamak é ainda caracterizado pela simetria azimutal (rotacional) e pelo uso da corrente de plasma para gerar a componente helicoidal do campo magnético, necessária para um equilíbrio estável.
\chapter{Fundamentação Matemática}
\section{Introdução ao Plasma}
%livro Bittencrt fundamentos of plasmas
%https://pt.wikipedia.org/wiki/Plasma
O plasma é um dos estados físicos da matéria, similar ao gás, no qual certa porção das partículas é ionizada
 (em física, núcleos atómicos, provenientes de átomos completamente ionizados, como os da radiação alfa, são habitualmente designados como partículas carregadas, a ionização é geralmente alcançada pela aplicação de elevadas energias aos átomos, seja através da aplicação de uma alta tensão elétrica ou por via de radiação de alta energia, um gás ionizado é chamado plasma).
 A premissa básica é que o aquecimento de um gás provoca a dissociação das suas ligações moleculares, convertendo-o em seus átomos constituintes. Além disso, esse aquecimento adicional pode levar à ionização (ganho ou perda de elétrons) dessas moléculas e dos átomos do gás, transformando-o em plasma contendo partículas carregadas (elétrons e íons positivos).
A presença de um número não desprezível de portadores de carga torna o plasma eletricamente condutor, de modo que ele responde fortemente a campos eletromagnéticos. O plasma, portanto, possui propriedades bastante diferentes das de sólidos, líquidos e gases e é considerado um estado distinto da matéria. Como o gás, o plasma não possui forma ou volume definidos, a não ser quando contido em um recipiente; diferentemente do gás, porém, sob a influência de um campo magnético ele pode formar estruturas como filamentos, raios e camadas duplas. Alguns plasmas comuns são as estrelas e placas de neônio. No universo, o plasma é o estado mais comum da matéria comum, a maior parte da qual se encontra no rarefeito plasma intergaláctico e em estrelas.
 Langmuir escreveu:

\begin{center}
\textit{    Com exceção das proximidades dos eletrodos, onde há bainhas contendo menos elétrons, o gás ionizado contém íons e elétrons em quantidades aproximadamente iguais, de modo que a carga espacial resultante é muito pequena. Nós usaremos o nome plasma para descrever esta região contendo cargas equilibradas de íons e elétrons.}
\end{center}

Também é definido como gás no qual uma fração substancial dos átomos está ionizada. Um processo simples de ionização é aquecer o gás para que aumente o impacto eletrônico, por exemplo. Desse modo, o gás se torna o plasma quando a adição de calor ou outra forma de energia faz com que um número significante de seus átomos libere alguns ou todos os seus elétrons. Esses átomos que perdem elétrons ficam ionizados, ou seja, com uma carga positiva resultante, e os elétrons separados de seus átomos ficam livres para se mover pelo gás, interagindo com outros átomos e elétrons.

Por apresentar-se num estado fluido similar ao estado gasoso, o plasma é comumente descrito ou como o "quarto estado de agregação da matéria" (os três primeiros sendo sólido, líquido e gasoso). Mas essa descrição não é muito precisa, pois a passagem de um gás para a forma de plasma não ocorre através de uma transição de fase bem definida, tal como nas transições do estado sólido para líquido e deste para gás. De todo modo, o plasma pode ser considerado como um estado distinto da matéria, caracterizado por possuir um número de partículas eletricamente carregadas que é suficiente para afetar suas propriedades e comportamento. Os plasmas são bons condutores elétricos, e suas partículas respondem fortemente a interações eletromagnéticas de grande alcance.

Quando o número de átomos ionizados é relativamente pequeno, a interação entre as partículas carregadas do gás ionizado é dominada por processos colisionais, ou seja, que envolvem principalmente colisões binárias entre elas. Quando o número de partículas carregadas é substancial, a interação entre as partículas carregadas é dominada por processos coletivos, ou seja, a dinâmica de cada uma delas é determinada pelos campos elétricos e magnéticos produzidos por todas as outras partículas carregadas do meio. Neste caso, o gás ionizado passa a ser denominado plasma

\subsection{Definição de um plasma} 

O plasma é livremente descrito como um meio eletricamente neutro de partículas positivas e negativas (isto é, a carga total de um plasma é aproximadamente zero). É importante notar que, embora não tenham limites, essas partículas não são "livres". Quando as cargas se movem, elas geram correntes elétricas com campos magnéticos e, como resultado, cada uma é afetada pelos campos das outras. Isto determina o comportamento coletivo com muitos graus de liberdade. 
\subsection{A aproximação de plasma} 	 
   Partículas carregadas devem estar suficientemente próximas, de modo que cada uma influencie muitas partículas carregadas na sua vizinhança, em vez de somente interagir com a mais próxima (esses efeitos coletivos são característicos de um plasma). A aproximação de plasma é válida quando o número de portadores de carga no interior da esfera de influência (chamada de esfera de Debye, cujo raio é o comprimento de Debye) de uma partícula em particular é maior do que uma unidade, para que haja comportamento coletivo das partículas carregadas. O número médio de partículas na esfera de Debye é representado pelo parâmetro de plasma "$\lambda$" (a letra grega lambda).
\subsection{Interações de volume} 
   O comprimento de Debye (definido acima) é pequeno se comparado ao tamanho físico do plasma. Este critério significa que as interações no interior do plasma são mais importantes do que nas bordas, onde podem ocorrer efeitos de fronteira. Quando este critério é obedecido, o plasma é praticamente neutro.
\subsection{Frequência de plasma} 
    A frequência dos elétrons do plasma (medindo a oscilação da densidade dos elétrons do plasma) é alta se comparada à frequência de colisões entre elétrons e partículas neutras. Quando esta condição é válida, as interações eletrostáticas predominam sobre os processos da cinética normal dos gases.
    Uma das manifestações fundamentais da propriedade coletiva do plasma é a oscilação de plasma. Seja um plasma em equilíbrio. Se se deslocar um elemento de cargas negativas da sua posição de equilíbrio, devido a interação coletiva de cargas da vizinhança aparece uma força restauradora proporcional ao deslocamento. O resultado é uma oscilação cuja frequência chamada frequência de plasma é expressa por:
\begin{equation}
    \omega_p = (\frac{4\pi n e^2}{m})^{\frac{1}{2}}
\end{equation}
A frequência de plasma é frequentemente utilizada como um meio de medir a densidade de um plasma. Existe uma relação entre a velocidade térmica, $v_\theta$, a frequência de plasma e o comprimento de Debye.
\begin{equation}
v_\theta = \lambda_D w_p
\end{equation}
Esta oscilação de plasma pode deixar de existir se a densidade de partículas neutras no plasma for aumentada, de modo que o tempo de colisões, $\tau$, entre a partícula carregada e a partícula neutra for inferior ao período de oscilação do plasma. Requer-se então:
\begin{equation}
 w_p\tau > 1 
 \end{equation} 
para existência desta oscilação.
\subsection{Grau de ionização}
	A ionização é necessária para o plasma existir. O termo "densidade do plasma" usualmente se refere à "densidade de elétrons", isto é, o número de elétrons livres por unidade de volume. O grau de ionização de um plasma é a proporção de átomos que perderam (ou ganharam) elétrons e é controlado principalmente pela temperatura. Mesmo um gás parcialmente ionizado, em que somente 1\% das partículas esteja ionizada, pode apresentar as características de um plasma, isto é, resposta a campos magnéticos e alta condutividade elétrica. O grau de ionização $\alpha$ é definido como $\alpha = n_i/(n_i + n_a)$, em que $n_i$ é a densidade de íons e $n_a$ é a densidade de átomos neutros. A densidade de elétrons está relacionada a ele pelo estado médio da carga $<Z>$ dos íons, sendo que $n_e = <Z> n_i$, em que ne é a densidade de elétrons.
\subsection{Temperaturas}

A temperatura do plasma é normalmente medida em kelvins ou elétron-volts e é, informalmente, uma medida da energia cinética térmica por partícula. Geralmente são necessárias temperaturas muito altas para sustentar a ionização, a qual é uma caraterística definidora de um plasma. O grau de ionização do plasma é determinado pela "temperatura do elétron" relativa ao potencial de ionização (e, com menos intensidade, pela densidade), numa relação chamada equação de Saha. Em baixas temperaturas, os íons e elétrons tendem a se recombinar para o seu estado ligado - átomos - e o plasma acaba se convertendo em um gás.

Na maioria dos casos os elétrons estão suficientemente próximos do equilíbrio térmico, de modo que sua temperatura é relativamente bem definida, mesmo quando há um desvio significativo de uma função de distribuição de energia maxwelliana, devido, por exemplo, a radiação ultravioleta, a partículas energéticas ou a campos elétricos fortes. Por causa da grande diferente de massa, os elétrons chegam ao equilíbrio termodinâmico entre si muito mais rapidamente do que com os íons ou átomos neutros. Por esta razão, a "temperatura do íon" pode ser muito diferente (normalmente menor) da "temperatura do elétron". Isto é especialmente comum em plasmas tecnológicos fracamente ionizados, cujos íons estão frequentemente próximos à temperatura ambiente.

Em função das temperaturas relativas dos elétrons, íons e partículas neutras, os plasmas são classificados como "térmicos" ou "não térmicos". Plasmas térmicos possuem elétrons e partículas pesadas à mesma temperatura, isto é, eles estão em equilíbrio térmico entre si. Plasmas não térmicos, por outro lado, possuem íons e átomos neutros a uma temperatura muito menor (normalmente temperatura ambiente), enquanto os elétrons são muito mais "quentes".

Um plasma é às vezes chamado de "quente" se ele está quase totalmente ionizado, ou "frio" se apenas uma pequena fração (por exemplo, 1\%) das moléculas do gás estão ionizadas, mas outras definições dos termos "plasma quente" e "plasma frio" são comuns. Mesmo em um plasma "frio", a temperatura do elétron é tipicamente de várias centenas de graus Celsius. Os plasmas utilizados na "tecnologia de plasma" ("plasmas tecnológicos") são normalmente frios neste sentido. 

\section{Função de Distribuição}
A função distribuição, $f$, é a densidade de partículas no espaço de fase de seis dimensões, três de posição e três de velocidade. Quando o plasma esta em equilíbrio termodinâmico a função de distribuição se torna Maxwelliana, conforme o teorema-H de Boltzmann. A distribuição Maxwelliana é dada por:
\begin{equation}
    f(\vv{v}) = n (\frac{m}{2\pi k T})^{\frac{3}{2}}exp(\frac{-mv^2}{kT})
\end{equation}
onde $n$ é a densidade média de partículas no espaço de configuração, $T$ é a temperatura do plasma, $m$ é a massa de uma partícula, $\vv{v}$ é a velocidade e $k$ a constante de Boltzmann.
Na maioria dos plasmas de laboratório o estudo é feito antes do plasma entrar em equilíbrio termodinâmico. Frequentemente estudos são feitos considerando que o plasma está em um certo equilíbrio, tais como elétrons em equilíbrio entre si a uma temperatura $T_e$ e íons em equilíbrio entre si a uma temperatura $T_i$, e investiga-se oque acontece com a distribuição a partir deste estado que se chama meta-equilíbrio.
\subsection{Temperatura e Outros Momentos da Função de Distribuição}
A função de distribuição é uma descrição microscópica de um plasma. Em contraste, uma descrição macroscópica de um plasma se faz pela especificação de valores médios das propriedades do plasma tais como, a densidade, a velocidade média, a pressão, a temperatura etc.
Existem relações entre a função de distribuição e estas grandezas macroscópicas, Assim:\\
densidade
\begin{equation}
    n_a = \int f_a d\vv{v}
\end{equation}
velocidade média
\begin{equation}
    \vv{V}_a = \int f_a \frac{d\vv{v}}{n_a}
\end{equation}
tensor de pressão
\begin{equation}
    \overline{\overline{P_a}} = \frac{m_a}{n_a}\int (\vv{v}-\vv{V}_a)(\vv{v}-\vv{V}_a) f_a d\vv{v}
\end{equation}
onde o índice $a$ indica espécie de partículas, tais como, elétrons, íons de massa $m_a$ e carga $q_a$, nêutrons de massa $m_a$, etc.
\subsection{O Gerador Magnetohidrodinâmico}
(Citação do livro \textbf{J. A. Bittencourt  (auth.)-Fundamentals of Plasma Physics-Springer New York (2004)} )\\
O gerador de energia magnetohidrodinâmico (MHD) converte a energia cinética de um plasma denso que flui através de um campo magnético para
energia elétrica. O seu princípio básico é bastante simples.
Suponha que um plasma flua com velocidade u (ao longo da direção x) através de um campo magnético aplicado B (na direção deles). A força de Lorentz $q (u x B)$ faz com que os íons se desloquem
para cima (na direção z) e os elétrons para baixo, de modo que, se
trodos são colocados nas paredes do canal e conectados a um circuito externo, então uma densidade de corrente $J = uE_{ind} = uu x B$ (onde u denota condutividade do plasma e $E_{ind}$ é o campo elétrico induzido) flui através do fluxo de plasma na direção z. Essa densidade de corrente, por sua vez, produz uma densidade de força $J x B$ (na direção x), que desacelera o plasma em fluxo. O resultado líquido é a conversão de parte da energia cinética do plasma que entra no gerador em energia elétrica que pode ser aplicada a
uma carga externa. Este processo tem a vantagem de operar sem a ineficiência de um ciclo de calor.
\chapter{Dedução Modelo de 2 Fluidos}
\section{Equação de Vlasov}
Uma maneira aproximada muito útil para descrever a dinâmica de um plasma é considerar que os movimentos das partículas do plasma são governados pelos campos externos aplicados mais os campos internos médios macroscópicos, suavizados no espaço e no tempo, devido à presença e movimento de todas as partículas de plasma. O problema de obter os campos eletromagnéticos internos macroscópicos (suavizados), no entanto, ainda é complexo e requer que uma solução auto-consistente seja obtida.
A equação de Vlasov é uma equação diferencial parcial que descreve a evolução temporal da função de \textit{distribuição} ($f_\alpha(\vv{r},\vv{v},t)$) no espaço de fase e que incorpora diretamente os campos eletromagnéticos internos macroscópicos suavizados. Pode ser obtido da equação de Boltzmann na ausencia de colisões \ref{eq: boltsmam}
\begin{equation}
\label{eq: boltsmam}
\frac{\partial f_\alpha(\vv{r},\vv{v},t)}{\partial t} +\vv{v} \nabla f_\alpha(\vv{r},\vv{v},t) + a \nabla_V f_\alpha(\vv{r},\vv{v},t) = 0
\end{equation}  
mas incluindo os campos suavizados internos no termo de força,
\begin{equation}
\label{eq: vlasov}
\frac{\partial f_\alpha(\vv{r},\vv{v},t)}{\partial t} +\vv{v} \nabla f_\alpha(\vv{r},\vv{v},t) + \frac{1}{m_\alpha}[\vv{F}_{ext}+q_0(\vv{E}_i+\vv{v} \times \vv{B}_i)] . \nabla_V f_\alpha(\vv{r},\vv{v},t)= 0
\end{equation}
Então \ref{eq: vlasov} é a equação de Vlasov.
Aqui $\vv{F}_{ext}$ representa a força externa, incluindo a força de Lorentz associada a quaisquer campos elétricos e magnéticos aplicados externamente, e $\vv{E}_i$ e $\vv{B}_i$ são campos elétricos e magnéticos suavizados internos devido à presença e movimento de todas as partículas carregadas dentro do plasma. Para que os campos eletromagnéticos macroscópicos internos $\vv{E}_i$ e $\vv{B}_i$ sejam consistentes com as densidades de carga e corrente macroscópicas existentes no próprio plasma, eles devem satisfazer as equações de Maxwell.
\begin{equation}
\nabla \vv{E}_i = \frac{\rho}{\epsilon_o}
\end{equation}
\begin{equation}
\nabla \vv{B}_i = 0
\end{equation}
\begin{equation}
\nabla \times \vv{E}_i = -\frac{\partial \vv{B}_i}{\partial t}
\end{equation}
\begin{equation}
\nabla \times \vv{B}_i = \mu_o (\vv{J} + \epsilon_0 \frac{\partial \vv{E}_i}{\partial t} )
\end{equation}
com a densidade de carga do plasma $\rho$ e a densidade de corrente do plasma $\vv{J}$ dada pelas expressões
\begin{equation}
\label{eq: rho}
\rho(\vv{r},t) = \sum_\alpha q_\alpha n_\alpha(\vv{r},t) = \sum_\alpha q_\alpha \int_v f_\alpha(\vv{r},\vv{v},t) d^3v
\end{equation}
\begin{equation}
\label{eq: densidadecorrente}
\vv{J}(\vv{r},t) = \sum_\alpha q_\alpha n_\alpha(\vv{r},t) \vv{u}_\alpha(\vv{r},t) = \sum_\alpha  q_\alpha \int_v \vv{v} f_\alpha(\vv{r},\vv{v},t) d^3v
\end{equation}
Aqui $ \vv{u}_\alpha(\vv{r},t)$ denota a velocidade média macroscópica de cada espécie de partícula $\alpha$ que é dada por
\begin{equation}
\vv{u}_\alpha(\vv{r},t) = \frac{1}{n_\alpha(\vv{r},t)} \int_v \vv{v} f_\alpha(\vv{r},\vv{v},t) d^3v
\end{equation}
Tais equações constituem um conjunto completo de equações a serem resolvidas simultaneamente. Por exemplo, em um procedimento iterativo assumindo valores aproximados iniciais para $\vv{E}_i(\vv{r}, t)$ e $\vv{B}_i(\vv{r}, t)$, a Eq. \ref{eq: vlasov} pode ser resolvida para produzir $f_\alpha(\vv{r},\vv{v},t)$ para as várias espécies diferentes de partículas. A utilização dos valores calculados em \ref{eq: rho} e \ref{eq: densidadecorrente} leva a valores para as densidades de carga e corrente ($\rho$ e  $\vv{J}$) no plasma, que podem ser substituídas nas equações de Maxwell e resolvidas para $\vv{E}_i(\vv{r}, t)$ e $\vv{B}_i(\vv{r}, t)$. Esses valores são então conectados novamente na equação de Vlasov (Eq. \ref{eq: vlasov}), e assim por diante, para obter uma solução autoconsistente para a função de distribuição de partículas individuais. Embora a equação de Vlasov não inclua explicitamente um termo de colisão no lado direito e, portanto, não leve em consideração colisões de curto alcance, ela não é tão restritiva quanto parece, já que uma parte significativa dos efeitos do interações de partículas já foram incluídas na força de Lorentz, através dos campos eletromagnéticos suavizados auto-consistentes internos.
\section{Momentos da distribuição}
Não é necessário resolver a equação de Boltzmann para a função de distribuição, a fim de calcular as variáveis macroscópicas de interesse físico. Estas variáveis macroscópicas estão relacionadas com os momentos da função de distribuição e as equações de transporte podem
ser obtido tomando os vários momentos da equação de Boltzmann. Os três primeiros momentos da equação de Boltzmann são obtidos pela multiplicação de $m_\alpha$, $m_\alpha \vv{v}_\alpha$ e $\frac{m_\alpha \vv{v}_\alpha}{2}$
2,
respectivamente, e integrando todo o espaço de velocidade. Seguindo estes procedimentos, obtêm-se as equações de conservação de massa, momento e energia. Seja $\chi(v)$ uma propriedade física das partículas no plasma. Agora multiplicamos cada termo da Eq \ref{eq: vlasov} por $\chi(v)$ e integramos a equação resultante sobre todo o espaço de velocidade para obter a equação como segue:

\begin{equation}
\label{eq: pit3}
\int_V \chi \frac{\partial f_\alpha}{\partial t} d^3 v + \int_V \chi \vv{v} \cdot \nabla f_\alpha d^3 v + \int_V \chi \vv{a} \cdot \nabla_v f_\alpha d^3 v = \int_V \chi (\frac{\delta f_\alpha}{\delta t})_{coll} d^3 v
\end{equation}
Resolvendo cada integral separadamente,

\begin{equation}
\label{eq: pit4}
\int_V \chi \frac{\partial f_\alpha}{\partial t} d^3 v  = \frac{\partial }{\partial t} (\int_V \chi f_\alpha d^3 v)-\int_V \frac{\partial \chi}{\partial t} f_\alpha d^3 v
\end{equation}
No entanto, desde que $\chi = \chi(\vv{v})$, sua derivada parcial em relação ao tempo é zero. Usando a definição de valores médios, o rendimento fica
\begin{equation}
\label{eq: pit5}
\int_V \chi \frac{\partial f_\alpha}{\partial t} d^3 v = \frac{\partial}{\partial t} [n_\alpha(\vv{r},t)<\chi>_\alpha]
\end{equation}

\begin{equation}
\label{eq: pit6}
\int_V \chi \vv{v} \nabla f_\alpha d^3 v = \nabla \cdot ( \int_V \chi \vv{v} f_\alpha d^3 v ) - \int_V \nabla \chi \cdot \vv{v} f_\alpha d^3 v - \int_V \chi  f_\alpha \nabla \cdot \vv{v}  d^3 v
\end{equation}
Como visto anteriormente, $\chi = \chi(\vv{v})$, então seu gradiente é zero e as variáveis $\vv{r}$, $\vv{v}$ e $t$ são independentes, então o divergente de $\vv{v}$ também é zero, resultando em

\begin{equation}
\label{eq: pit7}
\int_V \chi \vv{v} \cdot \nabla f_\alpha d^3 v = \nabla \cdot [n_\alpha(\vv{r},t)<\chi \vv{v}>_\alpha]
\end{equation}

\begin{equation}
\label{eq: pit8}
\int_V \chi \vv{a} \cdot \nabla_v f_\alpha d^3 v = \int_V \nabla_v \cdot (\vv{a} \chi f_\alpha)d^3 v - \int_V f_\alpha (\vv{a} \cdot \nabla_v \chi) d^3 v - \int_V \chi  f_\alpha (\nabla_v \cdot \vv{a})  d^3 v
\end{equation}
A primeira integral desaparece porque a função de distribuição deve desaparecer para $\pm \infty$. A última integral na eq \ref{eq: pit8} desaparece se assumirmos que

\begin{equation}
\label{eq: pit9}
\nabla_v \cdot \vv{a} = \frac{1}{m_\alpha} \nabla_v \cdot \vv{F} = 0
\end{equation}
isto é, se o componente de força $F_j$ for independente do componente de velocidade correspondente $v_j$, uma vez que $\nabla_v \cdot \vv{F} = \sum_j \frac{\partial F_j}{\partial v_j}$. Isto é verdade para a força devido a um campo magnético, $\vv{F} = q_\alpha \vv{v} \times \vv{B}$ porque $j$ também neste caso $F_j$ é independente de $v_i$.

\begin{equation}
\label{eq: pit10}
\int_V \chi \vv{a} \cdot \nabla_v f_\alpha d^3 v = -n_\alpha(\vv{r},t)<\vv{a} \cdot \nabla_v \chi>_\alpha
\end{equation}

\begin{equation}
\label{eq: pit11}
\int_V \chi ( \frac{\delta f_\alpha}{\delta t} )_{coll} d^3 v = [\frac{\delta}{\delta t}(n_\alpha(\vv{r},t)<\chi>_\alpha)]
\end{equation}

Combinando as eq \ref{eq: pit5}, eq \ref{eq: pit7}, eq \ref{eq: pit10} e eq \ref{eq: pit11} na eq \ref{eq: pit3} produzimos então,
\begin{equation}
\label{eq: pit12}
\frac{\partial }{\partial t}(n_\alpha(\vv{r},t)<\chi>_\alpha) + \nabla \cdot (n_\alpha(\vv{r},t)<\chi \vv{v} >_\alpha) - n_\alpha(\vv{r},t)<\vv{a} \cdot \nabla_v \chi>_\alpha = [\frac{\delta}{\delta t}(n_\alpha(\vv{r},t)<\chi>_\alpha)]
\end{equation}

\chapter{Cronograma}
O cronograma do trabalho encontra-se na Tabela \ref{tab-cronograma}, onde as etapas são:

\begin{enumerate}
\item Levantamento bibliográfico;
\item Analise inicial; 
\subitem Estudo do modelo de 2 fluidos para tokamak o \textit{breackdown};
\subitem Simplificações cabíveis do modelo para o Nova-Furg; 
\item Fase computacional; 
\subitem Linearir as equações relevantes;
\subitem Inserir as equações do modelo fisico no MATLAB;
\subitem Acoplar os solucionadores individuais correspondentes a cada equação;
\subitem Definir as condições de contorno e fronteira;
\item Análise dos resultados;
\subitem Relatório/síntese dos resultados.
\end{enumerate}

\begin{table}[h!]\begin{center}
	\caption{Cronograma}\label{tab-cronograma}
	\begin{tabular*}{\textwidth}{@{\extracolsep{\fill}} c c c c c c c c c c c c}
		\toprule
		& Etapa & mar. & abr. & maio & jun. & jul. & ago. & set. & out. & nov. &\\
		\midrule
		&   1   &   x  &   x  &      &      &   x  &      &      &  x   &      &\\
		&   2   &   x  &   x  &   x  &   x  &      &      &      &      &      &\\
		&   3   &      &      &      &      &   x  &   x  &   x  &   x  &   x  &\\
		&   4   &      &      &      &      &      &      &      &   x  &   x  &\\
		\bottomrule                             
	\end{tabular*}
\end{center}\end{table}

   
\chapter{Metodologia}
Iniciaremos esta seção apresentando uma breve dedução do modelo de 2 fluidos para o \textit{breackdown}, o qual sera usado para cumprir os objetivos desta proposta. Ver lista de notações em \ref{Listanot}.
\section{Equação de Vlasov}
De acordo com o modelo de fluidos para descrever a dinâmica de um plasma consideramos que os movimentos das partículas do plasma são governados pelos campos externos aplicados mais os campos internos médios macroscópicos que são suavizados no espaço e no tempo, porcausa da presença e movimento de todas as partículas de plasma. O problema de obter os campos eletromagnéticos internos macroscópicos, no entanto, ainda é complexo e requer que uma solução auto-consistente com as equações de Maxwell e a distribuião de velociades.
A equação de Vlasov é uma equação diferencial parcial usada para descrever a evolução temporal da função de \textit{distribuição} ($f_\alpha(\vv{r},\vv{v},t)$) no espaço das velocidades, posições e tempo, ou seja, descreve a evolução temporal da função de \textit{distribuição} no espaço de fase. E incorpora diretamente os campos eletromagnéticos internos macroscópicos suavizados. Pode ser obtida da equação de Boltzmann na ausencia de colisões Eq. \ref{eq: boltsmam} \cite[p. 193]{bittencourt}

\begin{equation}
\label{eq: boltsmam}
\frac{\partial f_\alpha(\vv{r},\vv{v},t)}{\partial t} +\vv{v} \nabla f_\alpha(\vv{r},\vv{v},t) + a \nabla_v f_\alpha(\vv{r},\vv{v},t) = 0
\end{equation}  
onde, $\vv{v}$ é a velocidade e os operadores diferenciais $\nabla_v$ e $\nabla$ em coordenanas cartesianas são 
$$\nabla_v f_\alpha(\vv{r},\vv{v},t) =  \frac{\partial f_\alpha(\vv{r},\vv{v},t)}{\partial v_x} + \frac{\partial f_\alpha(\vv{r},\vv{v},t)}{\partial v_y} + \frac{\partial f_\alpha(\vv{r},\vv{v},t)}{\partial v_z}$$  
$$\nabla f_\alpha(\vv{r},\vv{v},t) = \frac{\partial f_\alpha(\vv{r},\vv{v},t)}{\partial x} + \frac{\partial f_\alpha(\vv{r},\vv{v},t)}{\partial y} + \frac{\partial f_\alpha(\vv{r},\vv{v},t)}{\partial z}$$
mas incluindo na Eq. \ref{eq: boltsmam} os campos suavizados internos no termo de força, obtêm-se então a Eq. \ref{eq: vlasov} que é a equação de Vlasov.
\begin{equation}
\label{eq: vlasov}
\frac{\partial f_\alpha(\vv{r},\vv{v},t)}{\partial t} +\vv{v} \nabla f_\alpha(\vv{r},\vv{v},t) + \frac{1}{m_\alpha}[\vv{F}_{ext}+q_0(\vv{E}_i+\vv{v} \times \vv{B}_i)] . \nabla_v f_\alpha(\vv{r},\vv{v},t)= 0
\end{equation}

Aqui $\vv{F}_{ext}$ representa a força externa, incluindo a força de Lorentz associada a quaisquer campos elétricos e magnéticos aplicados externamente, onde $\vv{E}_i$ e $\vv{B}_i$ são campos elétricos e magnéticos suavizados internos causados pela presença e movimento de todas as partículas carregadas dentro do plasma. Os campos eletromagnéticos macroscópicos internos $\vv{E}_i$ e $\vv{B}_i$  devem satisfazer as equações de Maxwell, uma vez que precisam ser consistentes com as densidades de carga e corrente macroscópicas existentes no próprio plasma,
\begin{equation}
\label{eq: max1}
\nabla \vv{E}_i = \frac{\rho}{\epsilon_o}
\end{equation}
\begin{equation}
\label{eq: max2}
\nabla \vv{B}_i = 0
\end{equation}
\begin{equation}
\label{eq: max3}
\nabla \times \vv{E}_i = -\frac{\partial \vv{B}_i}{\partial t}
\end{equation}
\begin{equation}
\label{eq: max4}
\nabla \times \vv{B}_i = \mu_0 (\vv{J} + \epsilon_0 \frac{\partial \vv{E}_i}{\partial t} )
\end{equation}
Onde $\mu_0 = 4\pi \times 10^{-7}$ Newtons por Ampere ao quadrado  é a permeabilidade magnética do vácuo, $\epsilon_o = \frac{1}{\mu_0 c^2}$ é constante de permissividade do vácuo com $c$ sendo a velocidade da luz no vácuo e a densidade de carga do plasma $\rho$ é dada por
\begin{equation}
\label{eq: rho}
\rho(\vv{r},t) = \sum_\alpha q_\alpha n_\alpha(\vv{r},t) = \sum_\alpha q_\alpha \int_v f_\alpha(\vv{r},\vv{v},t) d^3v
\end{equation}
E a densidade de corrente de plasma $\vv{J}$ é dada por
\begin{equation}
\label{eq: densidadecorrente}
\vv{J}(\vv{r},t) = \sum_\alpha q_\alpha n_\alpha(\vv{r},t) \vv{u}_\alpha(\vv{r},t) = \sum_\alpha  q_\alpha \int_v \vv{v} f_\alpha(\vv{r},\vv{v},t) d^3v
\end{equation}
Aqui $ \vv{u}_\alpha(\vv{r},t)$ denota a velocidade média macroscópica de cada tipo de partícula $\alpha$, ver \ref{anexo1}. As equações Eq. \ref{eq: vlasov}, Eq. \ref{eq: max1}, Eq. \ref{eq: max2}, Eq. \ref{eq: max3}, Eq. \ref{eq: max4}, Eq. \ref{eq: rho} e Eq. \ref{eq: densidadecorrente} constituem um conjunto completo de equações a serem resolvidas ao mesmo tempo. Então, por exemplo, em um procedimento iterativo assumindo valores aproximados iniciais para $\vv{E}_i(\vv{r}, t)$ e $\vv{B}_i(\vv{r}, t)$, a Eq. \ref{eq: vlasov} pode ser resolvida para produzir $f_\alpha(\vv{r},\vv{v},t)$ para os tipos diferentes de partículas. A utilização dos valores calculados em Eq. \ref{eq: rho} e Eq. \ref{eq: densidadecorrente} leva a valores para as densidades de carga e corrente ($\rho$ e  $\vv{J}$) no plasma, que podem ser substituídas nas equações de Maxwell e resolvidas para $\vv{E}_i(\vv{r}, t)$ e $\vv{B}_i(\vv{r}, t)$. Esses valores são então introduzidos novamente na equação de Vlasov Eq. \ref{eq: vlasov}, e assim por diante, para obter uma solução autoconsistente para a função de distribuição de partículas individuais. Embora a equação Eq. \ref{eq: vlasov} não inclua explicitamente um termo de colisão no seu lado direito e portanto, não leve em consideração colisões de curto alcance, ela é mais geral doque parece, já que uma consideravel parte dos efeitos de interações de partículas já foram incluídas na força de Lorentz, por meio dos campos eletromagnéticos suavizados auto-consistentes internos. 

Não é necessário resolver a equação de Boltzmann para a função de distribuição, a fim de calcular as variáveis macroscópicas de interesse físico. Estas variáveis macroscópicas estão relacionadas com os momentos da função de distribuição pela equação geral de transporte, deduzina no apêndice \ref{eq: pit12}. As equações de transporte podem ser obtidas tomando os vários momentos da equação de Boltzmann. Neste trabalho sera deduzido os três primeiros momentos da equação de Boltzmann que são obtidos pela multiplicação de $m_\alpha$, $m_\alpha \vv{v}_\alpha$ e $\frac{m_\alpha \vv{v}^2_\alpha}{2}$, respectivamente, e então integrando todo o espaço de velocidade. Seguindo estas etapas, obtêm-se as equações de conservação de massa \ref{eq: pit13}, momento \ref{eq: pit23} e energia \ref{concenergia}.

\section{Equação de Conservação de Massa}

A equação de continuidade, nos garante que toda massa ganhada ou perdida no sistema é quantificada no termo $S_\alpha$.
Éssa equação pode ser obtida diretamente com a substituição na Eq. \ref{eq: pit12} do $\chi$ pela massa $m_\alpha$. Definindo a densidade de massa como $ \rho_{m\alpha} = n_\alpha m_\alpha$ temos

\begin{equation}
\label{eq: pit13}
\frac{\partial \rho_{m\alpha}}{\partial t} + \nabla \cdot (\rho_{m\alpha} \vv{u}_\alpha)  = S_\alpha 
\end{equation}

onde o termo de colisão, $S_\alpha = \left[\frac{\delta \rho_{m\alpha}}{\delta t}\right]_{coll}$, representa a taxa na qual partículas do tipo $\alpha$ são produzidas ou perdidas, por unidade de volume, como resultado de colisões.

\section{Equação de Conservação de Momento}
A equação de conservação de momento afirma que a taxa média de mudança de momento em funão do tempo em cada elemento $\alpha$ do fluido, é devida às forças externas aplicadas no fluido somadas a força de pressão do próprio fluido e também domadas as forças internas devido a colisões, dispersão e produção de partículas de plasma.
Para derivar a equação de conservação de momento, substituímos na Eq. \ref{eq: pit12} o $\chi (\vv{v}_\alpha)$ pelo momento $m_\alpha \vv{v}_\alpha$ das partículas do tipo $\alpha$. 
\begin{equation}
\label{eq: p2}
\frac{\partial }{\partial t}(n_\alpha(\vv{r},t)<m_\alpha \vv{v}_\alpha>_\alpha) + \nabla \cdot (n_\alpha(\vv{r},t)<m_\alpha \vv{v}_\alpha \vv{v} >_\alpha) - n_\alpha(\vv{r},t)<\vv{a} \cdot \nabla_v m_\alpha \vv{v}_\alpha>_\alpha = 
\end{equation}

\begin{equation*}
=[\frac{\delta}{\delta t}(n_\alpha(\vv{r},t)<m_\alpha \vv{v}_\alpha>_\alpha)]
\end{equation*}
Definindo $\vv{v}_\alpha = \vv{c}_\alpha + \vv{u}_\alpha$, onde $\vv{c}_\alpha$ é a velocidade média térmica das partículas e  $\vv{u}_\alpha$ é a velocidade média de movimento toroidal. Vamos tratar de cada termo da Eq. \ref{eq: p2} semaradamente.
Aplicando a definição de valor médio \ref{anexo1} e como $ \rho_{m\alpha} = n_\alpha m_\alpha$, ficamos com $\frac{\partial }{\partial t} (n_\alpha <\chi>_\alpha)=\frac{\partial }{\partial t} (n_\alpha <m_\alpha \vv{v}_\alpha>_\alpha)$ mas $\vv{c}_\alpha$ não influencia no valor médio, então ficamos com $\frac{\partial }{\partial t} (\rho_{m\alpha} \vv{u}_\alpha)$. Aplicando a regra da cadeia na derivada parcial obtemos
\begin{equation}
\label{eq: pit14}
\frac{\partial }{\partial t} (\rho_{m\alpha} \vv{u}_\alpha) = \vv{u}_\alpha \frac{\partial \rho_{m\alpha}}{\partial t}+\rho_{m\alpha} \frac{\partial \vv{u}_\alpha}{\partial t}
\end{equation} 
Substituindo $\chi$ pelo momento $m_\alpha \vv{v}_\alpha$ temos $\nabla \cdot (n_\alpha <\chi \vv{v}>_\alpha) =  \nabla \cdot (n_\alpha <m_\alpha \vv{v}_\alpha \vv{v}>_\alpha)$ e tirando $m_\alpha$ para fora da média, pois é constante temos $\nabla \cdot (n_\alpha m_\alpha < \vv{v} \cdot \vv{v}>_\alpha)$ nota-se que dentro do termo de média $ \vv{v}_\alpha$ é igual $\vv{v}$ pois o tipo de partícula $\alpha$ já esta representado na operação valor médio $< >_\alpha$ ficamos com $\nabla \cdot (\rho_{m\alpha} <\vv{v} \cdot \vv{v}>)$.  Aplicando a definição de valor médio \ref{anexo1} obtemos 
\begin{equation}
\label{eq: pit15}
\nabla \cdot (\rho_{m\alpha} <\vv{v} \cdot \vv{v}>) = \nabla \cdot ( \rho_{m\alpha}\vv{u}_\alpha \cdot \vv{u}_\alpha + \rho_{m\alpha}<\vv{c}_\alpha \cdot \vv{c}_\alpha>)
\end{equation} 
já que $\vv{c} \cdot \vv{u} = 0$ pois $\vv{c}$ vai em todas as direções então não contribui com a velocidade, e o valor médio de $\vv{v}_\alpha$ é por definição igual a $\vv{u}_\alpha$.
No termo seguinte, $n_\alpha(\vv{r},t)<\vv{a} \cdot \nabla_v m_\alpha \vv{v}>_\alpha$ rearanjamos os termos ficando com $n_\alpha(\vv{r},t)<(m_\alpha \vv{a} \cdot \nabla_v) \vv{v}>_\alpha$ observando que massa vezes aceleração é força, $\vv{F} = \vv{a} \cdot m_\alpha $, temos $n_\alpha(\vv{r},t) <(\vv{F} \cdot \nabla_v)\vv{v}>_\alpha$ que nos da
\begin{equation}
\label{eq: pit16}
n_\alpha(\vv{r},t) <(\vv{F} \cdot \nabla_v)\vv{v}>_\alpha=n_\alpha(\vv{r},t)<\vv{F}>_\alpha
\end{equation}
pois $(\vv{F} \cdot \nabla_v)\vv{v} = \vv{F} \cdot (\nabla_v\vv{v})$ e $\nabla_v\vv{v} = 1$. Definindo $R_\alpha$ igual a taxa de mudança de momento devido ao espalhamento onde
\begin{equation}
\label{eq: pit17}
R_\alpha = \left[\frac{\delta}{\delta t}(n_\alpha(\vv{r},t)<m_\alpha \vv{v}_\alpha>_\alpha)\right]=\left[\frac{\delta}{\delta t}(\rho_{m\alpha}\vv{u}_\alpha)\right]
\end{equation}
e  $\mathcal{R}_\alpha$ igual a taxa de mudança de momento devido à produção de partículas de plasma onde
\begin{equation}
\label{eq: pit18}
\mathcal{R}_\alpha = \left[\frac{\delta}{\delta t}(n_\alpha(\vv{r},t)<m_\alpha \vv{v}_\alpha>_\alpha)\right] = \left[\frac{\delta}{\delta t}(\rho_{m\alpha}\vv{u}_\alpha)\right]
\end{equation}
Substituindo Eq. \ref{eq: pit14}, Eq. \ref{eq: pit15}, Eq. \ref{eq: pit16}, Eq. \ref{eq: pit17}, Eq. \ref{eq: pit18} em Eq. \ref{eq: p2} obtemos
\begin{equation}
\label{eq: pit19}
\vv{u}_\alpha \frac{\partial \rho_{m\alpha}}{\partial t} + \rho_{m\alpha} \frac{\partial \vv{u}_\alpha}{\partial t}+ \nabla \cdot (\rho_{m\alpha} \vv{u}_\alpha \vv{u}_\alpha)+
\end{equation} 
\begin{equation*}
+\nabla \cdot (\rho_{m\alpha} <\vv{c}_\alpha \cdot \vv{c}_\alpha>)-n_\alpha<\vv{F}>_\alpha=R_\alpha + \mathcal{R}_\alpha
\end{equation*}
Definindo o tensor $\mathbb{P}_\alpha=\rho_{m\alpha} <\vv{c}_\alpha \cdot \vv{c}_\alpha>$ que representa a força por unidade de volume dentro do plasma devido aos movimentos aleatórios das partículas. Temos
\begin{equation}
\label{eq: pit20}
\nabla \cdot (\rho_{m\alpha} \vv{u}_\alpha \vv{u}_\alpha) =  \rho_{m\alpha}(\vv{u}_\alpha \cdot \nabla)\vv{u}_\alpha + [\nabla \cdot (\rho_{m\alpha} \vv{u}_\alpha)]\vv{u}_\alpha
\end{equation}
Fazendo força média igual a força de Lorentz temos 
\begin{equation}
\label{eq: pit20.1}
<\vv{F}>_\alpha = q_\alpha (\vv{E} + \vv{u}_\alpha \times \vv{B})
\end{equation} e substituindo as Eq. \ref{eq: pit20} e Eq. \ref{eq: pit20.1} em Eq. \ref{eq: pit19} teremos
\begin{equation}
\label{eq: pit21}
 \vv{u}_\alpha \frac{\partial \rho_{m\alpha}}{\partial t} + \rho_{m\alpha} \frac{\partial \vv{u}_\alpha}{\partial t}+ \rho_{m\alpha} (\vv{u}_\alpha \cdot \nabla)\vv{u}_\alpha+[\nabla \cdot (\rho_{m\alpha} \vv{u}_\alpha)]\vv{u}_\alpha  =
\end{equation}
\begin{equation*}
n_\alpha q_\alpha (\vv{E} + \vv{u}_\alpha \times \vv{B})-\nabla \cdot \mathbb{P}_\alpha + R_\alpha + \mathcal{R}_\alpha
\end{equation*}
rearanjando
\begin{equation}
\label{eq: pit22}
\rho_{m\alpha} \left[\frac{\partial \vv{u}_\alpha}{\partial t} + (\vv{u} \cdot \nabla)\vv{u}_\alpha \right]+ \vv{u}_\alpha \left[ \frac{\partial \rho_{m\alpha}}{\partial t} + \nabla \cdot (\rho_{m\alpha} \vv{u}_\alpha)  \right] = R_\alpha + \mathcal{R}_\alpha +n_\alpha(\vv{r},t) q_\alpha (\vv{E} + \vv{u}_\alpha \times \vv{B})-\nabla \cdot \mathbb{P}_\alpha 
\end{equation}
No entanto, o segundo termo no lado esquerdo da Eq. \ref{eq: pit22} é a equação de conservação de massa Eq. \ref{eq: pit13} então se não considerarmos as colisões que levam à geração ou perda de partículas, ou seja $S_\alpha = 0$, podemos simplificar a equação para
\begin{equation}
\label{eq: pit23}
\rho_{m\alpha} \left[\frac{\partial \vv{u}_\alpha}{\partial t} + (\vv{u} \cdot \nabla)\vv{u}_\alpha \right] =  R_\alpha + \mathcal{R}_\alpha +n_\alpha(\vv{r},t) q_\alpha (\vv{E} + \vv{u}_\alpha \times \vv{B})-\nabla \cdot \mathbb{P}_\alpha 
\end{equation}
A equação de concervação de momento estabelece a condição necessária para garantir a conservação do momento do sistema. O termo $-\nabla \cdot \mathbb{P}_\alpha $ representa a força causada pelas variações aleatórias nas velocidades de cada partícula que é exercida em um volume unitário do plasma. Esta força em cada unidade de volume também inclui forças associadas à pressão escalar e forças de corte tangenciais, que são as forças viscosas. No nosso  caso, o efeito da viscosidade é pequeno não sendo importante no nosso plasma. Os termos não-diagonais de $\mathbb{P}_\alpha$ podem ser desprezados.  Além disso, no caso em que quando a distribuição de velocidades de cada tipo de partícula é isotrópica (não à direção privilegiada, suas propriedades são as mesmas em qualquer direção). Os termos diagonais de $\mathbb{P}_\alpha$ são todos iguais e representam a pressão cinética escalar $p_\alpha$. Assim, desconsiderando os efeitos de viscosidade e considerando uma distribuição de velocidade isotrópica, temos $\mathbb{P}_\alpha = p_\alpha$, e a força por unidade de volume se torna $-\nabla \cdot \mathbb{P}_\alpha = -\nabla p_\alpha$, de acordo com a \cite[Cap 6]{bittencourt}.


\section{Equação de Conservação de Energia}
A equação de conservação de energia nos garante que toda energia cinética ganha ou perdida no sistema veio de colisões com outras partículas do sistema ou foi causado pelo campo elétrico $\vv{E}$.
Para derivar a equação de conservação de energia, substituímos  na Eq. \ref{eq: pit12} o $\chi (\vv{v}_\alpha)$ pela energia cinética $\frac{m_\alpha \vv{v}^2_\alpha}{2}$ das partículas do tipo $\alpha$ obtendo
\begin{equation}
\label{eq: p3}
\frac{\partial }{\partial t}(n_\alpha(\vv{r},t)<\frac{1}{2}m_\alpha v_{\alpha}^2>_\alpha) + \nabla \cdot (n_\alpha(\vv{r},t)<\frac{1}{2}m_\alpha v_{\alpha}^2 \vv{v} >_\alpha) - n_\alpha(\vv{r},t)<\vv{a} \cdot \nabla_v \frac{1}{2}m_\alpha v_{\alpha}^2>_\alpha = 
\end{equation}
\begin{equation*}
=[\frac{\delta}{\delta t}(n_\alpha(\vv{r},t)<\frac{1}{2}m_\alpha v_{\alpha}^2>_\alpha)]
\end{equation*}
O primeiro termo da Eq. \ref{eq: p3} sera simplificado trocando $ n_\alpha(\vv{r},t) m_\alpha$ por $\rho_{m\alpha}$ 
\begin{equation}
\label{eq: pit24}
\frac{\partial }{\partial t}(n_\alpha<\frac{m_\alpha \vv{v}^2_\alpha}{2}>_\alpha)=\frac{\partial }{\partial t}(\frac{\rho_{m\alpha}}{2}<\vv{v}^2_\alpha>_\alpha)
\end{equation}
abrindo $\vv{v}^2_\alpha$ a Eq. \ref{eq: pit24} fica
\begin{equation}
\label{eq: pit24.1}
\frac{\partial }{\partial t}(\frac{\rho_{m\alpha}}{2}<\vv{v}^2>_\alpha)=\frac{\partial }{\partial t}(\frac{\rho_{m\alpha}}{2}(\vv{u}_\alpha \cdot \vv{u}_\alpha) + \frac{\rho_{m\alpha}}{2}<\vv{c}_\alpha \cdot \vv{c}_\alpha>_\alpha)
\end{equation} 
Como a velocidade aleatória não tem sentido privelegiado, temos para qualquer partícula $|\vv{v}^2| = v_x^2 + v_y^2 + v_z^2$ e como são muitas partículas que se movem em direções aleatórias, os valores médios dos quadrados das componentes de suas velocidades são iguais, logo, $v_x^2 = \frac{1}{3} |\vv{v}^2|$ então da teoria cinética dos gases obtemos $p=\frac{nMv_x^2}{V}$ onde $M$ é a massa molar do gás de origem do plasma, $p$ é a pressão e $n$ o numero de moles. No nosso caso $\frac{nMv_x^2}{V} = \frac{nM <c^2_\alpha>}{3V} $ e a densidade de massa $ \rho_{m\alpha} = n_\alpha m_\alpha$ pode subtituir $\frac{nM}{V}$ nos deixando com $\rho_{m\alpha}<c^2_\alpha>=3p_\alpha$. 
Então a Eq. \ref{eq: pit24.1} simplifica para 
\begin{equation}
\label{eq: pit24.12}
\frac{\partial }{\partial t}(\frac{\rho_{m\alpha}}{2}(\vv{u}_\alpha \cdot \vv{u}_\alpha) + \frac{\rho_{m\alpha}}{2}<\vv{c}_\alpha \cdot \vv{c}_\alpha>_\alpha)=\frac{\partial }{\partial t}  \left( \frac{1}{2} 3p_\alpha+\frac{1}{2} \rho_{m\alpha}\vv{u}^2_\alpha \right) 
\end{equation} 
O segundo termo da Eq.\ref{eq: p3} pode ser organizado rearanjando termos e trocando $ n_\alpha(\vv{r},t) m_\alpha$ por $\rho_{m\alpha}$ que fica
\begin{equation}
\label{eq: pit24.10}
\nabla \cdot (n_\alpha(\vv{r},t)<\frac{1}{2}m_\alpha v_{\alpha}^2 \vv{v} >_\alpha) = \nabla \cdot \left[ \frac{\rho_{m\alpha}}{2}<(\vv{v} \cdot \vv{v})\vv{v}>_\alpha \right]
\end{equation}
Lembrando que $\vv{v} = \vv{u}_\alpha + \vv{c}_\alpha$ então  $<(\vv{v} \cdot \vv{v})\vv{v}>_\alpha$ pode ser expandido da seguinte maneira $
<[(\vv{u}_\alpha + \vv{c}_\alpha) \cdot (\vv{u}_\alpha + \vv{c}_\alpha)](\vv{u}_\alpha + \vv{c}_\alpha)> = <(\vv{u}^2_\alpha + \vv{c}^2_\alpha+2\vv{u}_\alpha \cdot \vv{c}_\alpha)(\vv{u}_\alpha + \vv{c}_\alpha)>$. Pode-se escrever a equação de conservação de energia de outra forma se substituir o fluxo de calor e o tensor de pressão cinética definidos como
\begin{equation}
\mathbb{P}_\alpha = \rho_{m\alpha}<\vv{c}_\alpha \cdot \vv{c}_\alpha> 
\end{equation}
onde $\mathbb{P}_\alpha$ é o tensor de pressão cinética
\begin{equation}
\vv{q}_\alpha = \frac{1}{2} \rho_{m\alpha} <c^2_\alpha  \vv{c}_\alpha>
\end{equation}
onde $\vv{q}_\alpha$ é o fluxo de calor
mas como vimos a pressão cinética é dada por $\rho_{m\alpha}<|\vv{c}^2|_\alpha>=3p_\alpha$ então a Eq. \ref{eq: pit24.10} fica
\begin{equation}
\label{eq: pit24.11}
\nabla \cdot \left[ \frac{\rho_{m\alpha}}{2}<(\vv{v} \cdot \vv{v})\vv{v}>_\alpha \right]= \nabla \cdot \left[ \frac{\rho_{m\alpha}}{2}\vv{u}_\alpha+\frac{1}{2}(3p_\alpha) + \mathbb{P}_\alpha \cdot \vv{u}_\alpha + \vv{q}_\alpha \right]
\end{equation}
simplificando \ref{eq: pit24.11} obtemos
\begin{equation}
\label{eq: pit24.13}
\nabla \cdot \left[ \frac{\rho_{m\alpha}}{2}\vv{u}_\alpha+\frac{1}{2}(3p_\alpha) + \mathbb{P}_\alpha \cdot \vv{u}_\alpha + \vv{q}_\alpha \right] = 
\end{equation} 

\begin{equation*}
\nabla \cdot \left[ \frac{1}{2}\rho_{m\alpha}|\vv{u}|^2_\alpha \vv{u}_\alpha) \right]+ \frac{3}{2}p_\alpha (\nabla \cdot \vv{u}_\alpha) +  \frac{1}{2}(\vv{u}_\alpha \cdot \nabla)(3p_\alpha)+\nabla \cdot (\mathbb{P} \cdot \vv{u}_\alpha) + \nabla \cdot \vv{q}_\alpha
\end{equation*}
Para o terceiro termo substituimos $\vv{a} =  \frac{\vv{F}}{m_\alpha}$ 
\begin{equation}
\label{eq: pit24.21}
n_\alpha(\vv{r},t) <\vv{a} \cdot \nabla_v \frac{m_\alpha \vv{v}^2_\alpha}{2}>_\alpha = n_\alpha(\vv{r},t)<\frac{\vv{F}}{m_\alpha} \cdot \nabla_v \left( \frac{m_\alpha \vv{v}^2}{2}\right)>_\alpha 
\end{equation}
resolvendo $\nabla_v ( \frac{m_\alpha \vv{v}^2}{2})$ obtemos $\frac{2m_\alpha \vv{v}}{2}$ e subtituindo em \ref{eq: pit24.2} e cancelando o $2m_\alpha$ que esta no denominador e dividendo ficamos com
\begin{equation}
\label{eq: pit24.4}
n_\alpha(\vv{r},t)<\frac{\vv{F}}{m_\alpha} \cdot \nabla_v \left( \frac{m_\alpha \vv{v}^2}{2}\right)>_\alpha = n_\alpha(\vv{r},t) <\vv{F} \cdot \vv{v}>_\alpha
\end{equation}
Substituindo as quantidades Eq. \ref{eq: pit24.12}, Eq. \ref{eq: pit24.13} e Eq. \ref{eq: pit24.4} na Eq. \ref{eq: p3} temos
\begin{equation}
\label{eq: p4}
\frac{\partial }{\partial t}  \left( \frac{1}{2} 3p_\alpha+\frac{1}{2} \rho_{m\alpha}\vv{u}^2_\alpha \right) +\nabla \cdot \left[ \frac{1}{2}\rho_{m\alpha}|\vv{u}|^2_\alpha \vv{u}_\alpha) \right]+ \frac{3}{2}p_\alpha (\nabla \cdot \vv{u}_\alpha) +  \frac{1}{2}(\vv{u}_\alpha \cdot \nabla)(3p_\alpha)+ 
\end{equation}
\begin{equation*}
\nabla \cdot (\mathbb{P} \cdot \vv{u}_\alpha) + \nabla \cdot \vv{q}_\alpha-n_\alpha(\vv{r},t) <\vv{F} \cdot \vv{v}>_\alpha =[\frac{\delta}{\delta t}(n_\alpha(\vv{r},t)<\frac{1}{2}m_\alpha v_{\alpha}^2>_\alpha)]
\end{equation*}

Definindo $M_\alpha$ que representa a taxa de mudança de densidade de energia devido a colisões
\begin{equation}
M_\alpha = \frac{3}{2} \left(  \frac{\partial p_\alpha}{\partial t}  \right) + \frac{\partial}{\partial t} \left[ \frac{1}{2}\rho_{m\alpha}|\vv{u}|^2_\alpha \right]+ \nabla \cdot \left[\frac{1}{2}\rho_{m\alpha}<(\vv{v} \cdot \vv{v})\vv{v}>_\alpha \right] -  n_\alpha(\vv{r},t)<\vv{F} \cdot \vv{v}>_\alpha
\end{equation}

Usando $\frac{D}{Dt} = \frac{\partial }{\partial t} + \vv{u}_\alpha \cdot \nabla$ que é a derivada total do tempo obtemos

\begin{equation}
\label{eq: p5}
\frac{D}{Dt} \left( \frac{3}{2}p_\alpha \right) + \left(\frac{3}{2}p_\alpha \right)  \nabla \cdot \vv{u}_\alpha +  \frac{\partial }{\partial t} \left(\frac{\rho_{m\alpha}}{2} |\vv{u}|^2_\alpha \right) +\nabla \cdot \left[ \frac{1}{2}\rho_{m\alpha}|\vv{u}|^2_\alpha \vv{u}_\alpha \right]+  = 
\end{equation}
\begin{equation*}
\nabla \cdot (\mathbb{P} \cdot \vv{u}_\alpha) + \nabla \cdot \vv{q}_\alpha-n_\alpha(\vv{r},t) <\vv{F} \cdot \vv{v}>_\alpha = M_\alpha
\end{equation*}
Reescrevendo os termos $\frac{\partial }{\partial t} \left(\frac{\rho_{m\alpha}}{2} |\vv{u}|^2_\alpha \right) +\nabla \cdot \left[ \frac{1}{2}\rho_{m\alpha}|\vv{u}|^2_\alpha \vv{u}_ \alpha \right]$ em função da derivada total $\frac{D}{Dt}$ ficamos com
\begin{equation}
\frac{\partial }{\partial t} \left(\frac{\rho_{m\alpha}}{2}\vv{u}_\alpha \cdot \vv{u}_\alpha \right) +\nabla \cdot \left[ \frac{1}{2}\rho_{m\alpha}(\vv{u}_\alpha \cdot \vv{u}_ \alpha) \vv{u}_\alpha \right] = 
\end{equation}

\begin{equation*}
= \frac{1}{2}|\vv{u}|^2_\alpha \frac{\partial \rho_{m\alpha} }{\partial t} + \rho_{m\alpha} \vv{u}_\alpha \cdot \frac{\partial \vv{u}_\alpha}{\partial t} +
\frac{1}{2}|\vv{u}|^2_\alpha \nabla \cdot \left[ \rho_{m\alpha}  \vv{u}_\alpha \right] + \rho_{m\alpha}  \vv{u}_\alpha  \cdot \left[ (\vv{u}_\alpha \cdot \nabla )\vv{u}_\alpha  \right] = 
\end{equation*}

\begin{equation*}
= \frac{1}{2}|\vv{u}|^2_\alpha \left[  \frac{\partial \rho_{m\alpha} }{\partial t} +  \nabla \cdot (\rho_{m\alpha} \vv{u}_\alpha)   \right] + \rho_{m\alpha}  \vv{u}_\alpha \cdot \frac{\partial \vv{u}_\alpha}{\partial t}
\end{equation*}
 Usando as equações de concervação de massa e momento, Eq. \ref{eq: pit13} e Eq. \ref{eq: pit23} 
\begin{equation}
\label{eq: p6}
 \frac{1}{2}|\vv{u}|^2_\alpha S_\alpha + n_\alpha(\vv{r},t)\vv{u}_\alpha<F>_\alpha-\vv{u}_\alpha \cdot (\nabla \cdot \mathbb{P})+\vv{u}_\alpha \cdot \vv{A}_\alpha-|\vv{u}|^2_\alpha S_\alpha
 \end{equation} 
 Substituindo em Eq.  \ref{eq: p6} o resultado de Eq. \ref{eq: p5}
 \begin{equation}
\label{eq: p5}
\frac{D}{Dt} \left( \frac{3}{2}p_\alpha \right) + \left(\frac{3}{2}p_\alpha \right)  \nabla \cdot \vv{u}_\alpha +  \frac{\partial }{\partial t} \left(\frac{\rho_{m\alpha}}{2} |\vv{u}|^2_\alpha \right) +\nabla \cdot \left[ \frac{1}{2}\rho_{m\alpha}|\vv{u}|^2_\alpha \vv{u}_\alpha \right]+  = 
\end{equation}
\begin{equation*}
\nabla \cdot (\mathbb{P} \cdot \vv{u}_\alpha) + \nabla \cdot \vv{q}_\alpha-n_\alpha(\vv{r},t) <\vv{F} \cdot \vv{v}>_\alpha = M_\alpha
\end{equation*}
Definindo  $Q_\alpha$ como sendo a taxa de variação da densidade de energia devido ao espalhamento onde
\begin{equation}
Q_\alpha = \left[ \frac{\delta}{\delta t}(n_\alpha<\frac{m_\alpha \vv{v}^2_\alpha}{2}>_\alpha) \right] = \left[ \frac{\delta}{\delta t}\left( \frac{1}{2} \rho_{m\alpha} u^2_\alpha \right) \right]
\end{equation}
E definindo $\mathcal{E}_\alpha$ que representa a taxa de variação da densidade de energia devido a produção de partículas de plasma onde 
\begin{equation}
\mathcal{E}_\alpha = \left[ \frac{\delta}{\delta t}(n_\alpha<\frac{m_\alpha \vv{v}^2_\alpha}{2}>_\alpha) \right] = \left[ \frac{\delta}{\delta t}\left( \frac{1}{2} \rho_{m\alpha} u^2_\alpha \right) \right]
\end{equation}
Desta forma a Eq. \ref{eq: p4} torna-se então
\begin{equation}
\label{concenergia}
\frac{3}{2} \left( \frac{\partial p_\alpha}{\partial t} + \vv{u}_\alpha \cdot \nabla p_\alpha\right)+ \frac{3}{2} p_\alpha (\nabla \cdot \vv{u}_\alpha)+(\mathbb{P}_\alpha \cdot \nabla)\cdot \vv{u}_\alpha + \nabla \cdot \vv{q}_\alpha =  
\end{equation}
\begin{equation*}
= Q_\alpha + \mathcal{E}_\alpha-\vv{u}_\alpha \cdot \vv{R}_\alpha - \vv{u}_\alpha \cdot \mathcal{R}_\alpha+\frac{1}{2}u^2_\alpha S_\alpha
\end{equation*}
uma vez que $\frac{\partial }{\partial t}  \left( \frac{1}{2} 3p_\alpha+\frac{1}{2} \rho_{m\alpha}\vv{u}^2_\alpha \right) = \frac{3}{2} \left( \frac{\partial p_\alpha}{\partial t} + \vv{u}_\alpha \cdot \nabla p_\alpha\right)$
\section{Simplificações}
Primeiramente assumimos $n_e = n_i = n_0$ onde $n_0$ é a densidade eletronica em $t = 0$. Para modelar o Breakdown do plasma no tokamak Nova-Furg, usaremos um modelo de dois fluidos. Assumiremos que o plasma é composto por um fluido de elétrons e um único fluido ionizado, ambos sendo incompressíveis, $\nabla \vv{u}_{\alpha} = 0$,  um fluido ideal que não possua viscosidade $\mathbb{P}_{\alpha} = p_{\alpha}$, quase neutro $n_e = n_i = n$, e adiabático $\nabla \vv{q}_{\alpha} = 0$, ou seja, as escalas de tempo envolvidas são muito curtas para ocorrer a difusão de calor, portanto o processo sera considerdo adiabático e o transporte de energia é predominantemente convectivo. Para melhorar a estabilidade numérica, vamos considerar o fluxo total de partículas a ser composto por um termo convectivo e difusivo, isto é, $\Gamma=n\vv{u}_{\alpha}-D_{\alpha} \nabla n$ com $D_e = D_i = D$ sendo o coeficiente de difusão de partículas. Além do que, além do mais,
Vamos supor também que o gás neutro esteja em repouso e à temperatura ambiente. Portanto, os termos de momento e fonte de energia devido à produção de partículas de plasma $\mathcal{R}_{\alpha}$ e $\mathcal{E}_{\alpha}$ não contribuem para o momento e conteúdo térmico do plasma. Com essas suposições,
o conjunto de equações consideradas neste modelo são:

\begin{equation}
\frac{\partial n}{\partial t} + \nabla \cdot (n \vv{u}_\alpha) = \frac{S_\alpha}{m_\alpha}+D\nabla^2 n
\end{equation}
\begin{equation}
m_\alpha n \left[ \frac{\partial \vv{u}_\alpha}{\partial t} + (\vv{u}_\alpha \cdot \nabla)\vv{u}_\alpha \right] =  q_\alpha n (\vv{E}+\vv{u}_-\alpha \times \vv{B}) - \nabla p_\alpha + \vv{R}_\alpha - \vv{u}_\alpha S_\alpha 
\end{equation}
\begin{equation} 
\frac{3}{2} \left( \frac{\partial p_\alpha}{\partial t} + \vv{u}_\alpha \cdot \nabla p_\alpha \right) =Q_\alpha -\vv{u}_\alpha \cdot \vv{R}_\alpha \frac{1}{2} u^2_\alpha S_\alpha
\end{equation}
Aqui, vamos supor que o termo de troca de momento $\vv{R}_\alpha$ pode ser modelado por
\begin{equation}
\vv{R}_\alpha = m_\alpha n \sum_\rho  \mu_{\alpha \rho}(\vv{u}_\alpha - \vv{u}_\rho)
\end{equation}
e portanto,
\begin{equation}
\vv{R}_e = m_e n \nu_{en}\vv{u}_e+ne\eta \vv{J}
\end{equation}
\begin{equation}
\vv{R}_i = m_i n \nu_{in}\vv{u}_i+ne\eta \vv{J}
\end{equation}
com $\eta$ sendo a resistividade paralela do Spitzer, e que os termos de troca de elétrons e de íons podem ser modelados por
\begin{equation}
\vv{Q}_e = -\vv{Q}_i = -\frac{3ne^2}{m_i} \eta (p_e-p_i)
\end{equation}
\\
O conjunto final de equações que devem ser resolvidas é então\\
\begin{center}
{\large equações de conservação de massa para elétrons e íons}
\end{center}
\begin{equation}
\label{eq: pit1}
\frac{\partial n}{\partial t} + \nabla.(n \vv{u}_e) = \frac{S_e}{m_e}+D\nabla^2 n
\end{equation}
\begin{equation} 
\label{eq: pit2}
\frac{\partial n}{\partial t} + \nabla.(n \vv{u}_i) = \frac{S_i}{m_i}+D\nabla^2 n
\end{equation}
\begin{flushleft}
\begin{center}
{\large equações de conservação de momento para elétrons e íons}
\end{center}
\end{flushleft}
\begin{equation}
[\frac{\partial \vv{u}_e}{\partial t} + (\vv{u}_e . \nabla)\vv{u}_e] = - \frac{e}{m_e} (\vv{E}+\vv{u}_e \times \vv{B}) - \frac{\nabla p_e}{m_e n} - (\nu_{en}+\frac{S_e}{m_e n})\vv{u}_e + \frac{e}{m_e}\eta \vv{J} 
\end{equation}
\begin{equation}
[\frac{\partial \vv{u}_i}{\partial t} + (\vv{u}_i . \nabla)\vv{u}_i] = - \frac{e}{m_i} (\vv{E}+\vv{u}_i \times \vv{B}) - \frac{\nabla p_i}{m_i n} - (\nu_{in}+\frac{S_i}{m_i n})\vv{u}_i + \frac{e}{m_i}\eta \vv{J} 
\end{equation}
\begin{center}
{\large equações de conservação de energia para elétrons e íons}
\end{center}
\begin{equation}
\frac{3}{2} (\frac{\partial p_e}{\partial t} + \vv{u}_e . \nabla p_e) = -\frac{3ne^2}{m_i} \eta(p_e-p_i)-ne\eta \vv{u}_e.\vv{J}+(2 \nu_{en} + \frac{S_e}{m_e n} )\frac{1}{2}m_en{u_e}^2
\end{equation}

\begin{equation}
\frac{3}{2} (\frac{\partial p_i}{\partial t} + \vv{u}_i . \nabla p_i) = -\frac{3ne^2}{m_i} \eta(p_e-p_i)-ne\eta \vv{u}_i.\vv{J}+(2 \nu_{in} + \frac{S_i}{m_i n} )\frac{1}{2}m_in{u_i}^2
\end{equation}

Suponhamos agora que, durante a fase inicial, os íons podem ser considerados em repouso ($\vv{u}_i = 0$) e, portanto, podemos remover a equação de conservação do momento iônico. Vamos também adicionar as equações Eq. \ref{eq: pit1} e Eq. \ref{eq: pit2}. Para isso, teremos que definir um termo de fonte de partículas que explique o número de partículas em si em vez da massa de partículas: $\zeta_\alpha = S_\alpha / m_\alpha$. Note que para um plasma ionizado sozinho, podemos dizer que $\zeta_e = \zeta_i = \zeta = n(\nu_{ion} - \nu_{loss})$. Também assumiremos que $\nu_{ion} = \alpha_T u_e = \alpha_T J / (ne)$ e $v_{loss} = u_{e, ||} / L_{eff} = \vv{J}.\vv{B} (Bne)$. Aqui, $\alpha_T$ é o primeiro coeficiente de Townsend. Com todas essas suposições, o conjunto de equações simplifica para
\begin{equation}
\frac{\partial n}{\partial t} = \nu_{ion} - \nu_{loss}+D\nabla^2n
\end{equation}

\begin{equation}
m_e[\frac{\partial \vv{u}_e}{\partial t} + (\vv{u}_e . \nabla)\vv{u}_e] = - e (\vv{E}+\vv{u}_e \times \vv{B}) - \frac{\nabla p_e}{n} -(\nu_{en}+\nu_{ion}-\nu_{loss})m_e \vv{u}_e+ e\eta \vv{J} 
\end{equation}

\begin{equation}
\frac{3}{2} (\frac{\partial p_e}{\partial t} + \vv{u}_e . \nabla p_e) = -\frac{3ne^2}{m_i} \eta(p_e-p_i)-ne\eta \vv{u}_e.\vv{J}+(2 \nu_{en} + \nu_{ion} - \nu_{loss} )\frac{1}{2}m_en{u_e}^2
\end{equation}

\begin{equation}
\frac{\partial p_i}{\partial t} = \frac{2ne^2}{m_i}\eta(p_e-p_i)
\end{equation}
Vamos agora reescrever este conjunto de equações em termos da densidade de corrente
\begin{equation}
\frac{\partial n}{\partial t} = \nu_{ion} - \nu_{loss}+D\nabla^2n
\end{equation}

\begin{equation}
\frac{\partial \vv{J}}{\partial t} =  \frac{ne^2}{m_e} \vv{E} -\vv{J}(\nu_{in}+\nu_{en}+\nu_{ion}-\nu_{loss}) -\frac{e}{m_e}\vv{J} \times \vv{B}+\frac{e}{m_e}\nabla p_e 
\end{equation}

\begin{equation}
\frac{\partial p_e}{\partial t} = \frac{3}{2}(1+\frac{2 \nu_{en} + \nu_{ion} - \nu_{loss}}{2\nu_{ei}})\eta J^2  -\frac{2ne^2}{m_i} \eta (p_e-p_i)
\end{equation}

\begin{equation}
\frac{\partial p_i}{\partial t} = \frac{2ne^2}{m_i}\eta(p_e-p_i)
\end{equation}

Para entrarmos na parte computacional ainda é nessesário incluir as esquações para os campos magnéticos e separar seus componentes em um componente de campo toroidal e outro de campo poloidal com um padrão quadrupolo. Isso sera feito no TCC 2, juntamente com uma mudança de coordenadas, para um sistema de coordenadas cílindricas que se fecha em si mesmo, explicando de outra forma, é como se no interior de um toroide (com as dimenções internas da câmera de vácuo do tokamak Nova-Furg), cada seção reta circular fosse graduada com um sistema de coordenadas polares ($R,\theta$) e com um angulo $\varphi$ que representa o quanto a seção reta do toride se deslocou da origem onde $\varphi=0$). Sera feito a adptação do modelo de dois fluidos para as coordenadas ($R,\theta,\varphi$) e também a implementação do modelo no MATLAB, definição das condições de contorno e fronteira e realização da simulação. 


\chapter{Resultados esperados}
Espero neste trabalho obter dados de simulação querentes com os medidos nos disparos do tokamak, para reforçar a válidade do modelo de dois fluidos para o \textit{breackdown}. Podendo então usar o modelo de dois fluidos para prever os resultados de cada experiemento na maquina. E se possível calcular quais os melhores parâmetros para operar o tokamak Nova-Furg.

\chapter{Conclusão}
Como resultados da fase teórica, logo no ínicio do trabalho vimos que existe claras diferenças entre métodos de fluidos para o plasma e métodos cinéticos e para o nosso caso a melhor opção são os modelos de fluidos. Constatamos que a equação de Vlasov, Eq. \ref{eq: vlasov}, não é tão restritiva quanto parece uma vez que uma parte muito siguinificativa das interações entre partículas são consideradas nos campos elétromagneticos  internos suavizados $\vv{E}_i$ e $\vv{B}_i$ causados pela presença e movimento de todas as partículas carregadas dentro do plasma. Outro importante resultado obtido neste trabalho é o conjunto das 6 equações de conservaão dos momentos da equação de Boltzmann Eq. \ref{eq: boltsmam}, para cada momento são duas equações, uma para o fluido de elétrons e outra para o fluido de íons. Mas como vimos nas simplificações os íons podem ser considerados em repouso durante a fase inicial ou seja $\vv{u}_i = 0$, então pode-se remover a equação de conservação do momento iônico na faze de brackdown. Uma vez que a massa dos íons é muito maior que a massa dos elétrons implicando em uma mobilidade de íons muito menor. 
%\newpage
\appendix  
\chapter{O modelo de Townsend}
\label{Townsend}
O modelo de Townsend assume que os campos elétricos acionados externamente são predominantes no dispositivo. Tal suposição é verdadeira para dispositivos com gás neutro de baixa pressão e curto espaço entre os eletrodos, porque a quantidade de carga espacial local produzida durante o inicio do avalanche de elétrons é desprezível devido à pequena taxa de crescimento, $\alpha_T$. Portanto as características da avalanche de elétrons em dispositivos de baixa pressão de gás dependendem dos campos elétricos externos e podem ser descritas pelo modelo de Townsend. 

Assumindo que, devido à sua maior massa, os íons são estacionários, e os elétrons livres são responsáveis pela ionização.  O aumento da densidade eletrônica, $n_e$, é proporcional à diferença entre a taxa de ionização, $\nu_{ionz}$ (geração de elétrons) e a taxa de perda de elétrons, $\nu_{loss}$
\begin{equation}
\dfrac{dn_e}{dt} = (\nu_{ionz}-\nu_{loss})n_e
\label{cac0}
\end{equation}
portanto, nesta fase, a densidade eletronica $n_e(t)$ pode ser expressa como
\begin{equation}
n_e(t) = n_{e0} e^{(\nu_{ionz}-\nu_{loss})t}
\label{cac1}
\end{equation}
onde $t$ descreve o tempo, $n_{e0}$ é a densidade de elétrons em $t=0$ o \textit{breackdown} ocorre quando a taxa de geração de elétrons excede a taxa de perda de elétrons. Estes elétrons então são acelerados pelo campo elétrico toroidal e alcançam a velocidade de desvio (\textit{drift velocity}). Observe que a Eq. \ref{cac1} é válida somente quando o grau de ionização permanece pequeno, deste modo as colisões de elétrons com as partículas neutras predominam sobre as colisões Coulombianas. Para modelar o processo de ionização, vamos escrever a taxa de ionização em termos do primeiro coeficiente de Townsend e a velocidade de desvio do elétron 
\begin{equation}
\nu_{ionz} = \alpha_T u_{D,||}
\end{equation} 
onde $u_{D, ||}$ é a velocidade de desvio, $\alpha_T$ é o primeiro coeficiente de Townsend que é dado por
\begin{equation}
\alpha_T = A p_n e^{B \frac{p_n}{E_\Theta}}
\end{equation}

Aqui, $A$ e $B$ são constantes que dependem do gás de trabalho, $p_n$ é a pressão de gás neutro e  $E_\Theta$ é a magnitude do campo elétrio externo. A taxa de perda de elétrons devido ao seu movimento ao longo das linhas do campo magnético pode ser expressa por
\begin{equation}
\nu_{loss} = \frac{u_{D,||}}{L_{eff}}
\end{equation}
onde $L_{eff}$ é a distância média percorida pelo elétron antes de se chocar com a parede. Fazendo o lado direito da Eq. \ref{cac0} ir para zero, cria-se uma condição para o início do \textit{breackdown}
\begin{equation}
A p_n e^{B \frac{p_n}{E_\Theta}} = \frac{1}{L_{eff}}
\end{equation}
Esta equação mostra que um \textit{breackdown} bem-sucedido em um tokamak depende da escolha da pressão do gás neutro, da intensidade do campo elétrico toroidal e do comprimento efetivo da conexão ($L_{eff}$), que depende da configuração do campo poloidal durante a fase de partida. Existe então um valor mínimo do campo elétrico toroidal induzido para uma dada pressão de gás neutro e $L_{eff}$ para que se possa obter um \textit{breackdown}. Uma explicação detalhada para a modelagem da teoria de Townsend pode ser encontrada em \cite{yoo2014ohmic}
 



\chapter{Momentos da distribuição}
 Seja $\chi(v)$ uma propriedade física das partículas no plasma. Agora multiplicamos cada termo da Eq. \ref{eq: vlasov} por $\chi(v)$ e integramos a equação resultante sobre todo o espaço de velocidade para obter a equação como segue:

\begin{equation}
\label{eq: pit3}
\int_V \chi \frac{\partial f_\alpha}{\partial t} d^3 v + \int_V \chi \vv{v} \cdot \nabla f_\alpha d^3 v + \int_V \chi \vv{a} \cdot \nabla_v f_\alpha d^3 v = \int_V \chi (\frac{\delta f_\alpha}{\delta t})_{coll} d^3 v
\end{equation}
Resolvendo cada integral separadamente,

\begin{equation}
\label{eq: pit4}
\int_V \chi \frac{\partial f_\alpha}{\partial t} d^3 v  = \frac{\partial }{\partial t} (\int_V \chi f_\alpha d^3 v)-\int_V \frac{\partial \chi}{\partial t} f_\alpha d^3 v
\end{equation}
No entanto, desde que $\chi = \chi(\vv{v})$, sua derivada parcial em relação ao tempo é zero. Usando a definição de valores médios (Anexo. \ref{anexo1}), o rendimento fica
\begin{equation}
\label{eq: pit5}
\int_V \chi \frac{\partial f_\alpha}{\partial t} d^3 v = \frac{\partial}{\partial t} [n_\alpha(\vv{r},t)<\chi>_\alpha]
\end{equation}

\begin{equation}
\label{eq: pit6}
\int_V \chi \vv{v} \nabla f_\alpha d^3 v = \nabla \cdot \left( \int_V \chi \vv{v} f_\alpha d^3 v \right) - \int_V \nabla \chi \cdot \vv{v} f_\alpha d^3 v - \int_V \chi  f_\alpha \nabla \cdot \vv{v}  d^3 v
\end{equation}
Como visto anteriormente, $\chi = \chi(\vv{v})$, então seu gradiente é zero e as variáveis $\vv{r}$, $\vv{v}$ e $t$ são independentes, então o divergente de $\vv{v}$ também é zero, resultando em

\begin{equation}
\label{eq: pit7}
\int_V \chi \vv{v} \cdot \nabla f_\alpha d^3 v = \nabla \cdot [n_\alpha(\vv{r},t)<\chi \vv{v}>_\alpha]
\end{equation}

\begin{equation}
\label{eq: pit8}
\int_V \chi \vv{a} \cdot \nabla_v f_\alpha d^3 v = \int_V \nabla_v \cdot (\vv{a} \chi f_\alpha)d^3 v - \int_V f_\alpha (\vv{a} \cdot \nabla_v \chi) d^3 v - \int_V \chi  f_\alpha (\nabla_v \cdot \vv{a})  d^3 v
\end{equation}
A primeira integral desaparece porque a função de distribuição deve desaparecer para $\pm \infty$. A última integral na Eq. \ref{eq: pit8} desaparece se assumirmos que

\begin{equation}
\label{eq: pit9}
\nabla_v \cdot \vv{a} = \frac{1}{m_\alpha} \nabla_v \cdot \vv{F} = 0
\end{equation}
isto é, se o componente de força $F_j$ for independente do componente de velocidade correspondente $v_j$, uma vez que $\nabla_v \cdot \vv{F} = \sum_j \frac{\partial F_j}{\partial v_j}$. Isto é verdade para a força devido a um campo magnético, $\vv{F} = q_\alpha \vv{v} \times \vv{B}$ porque $j$ também neste caso $F_j$ é independente de $v_i$.

\begin{equation}
\label{eq: pit10}
\int_V \chi \vv{a} \cdot \nabla_v f_\alpha d^3 v = -n_\alpha(\vv{r},t)<\vv{a} \cdot \nabla_v \chi>_\alpha
\end{equation}

\begin{equation}
\label{eq: pit11}
\int_V \chi \left( \frac{\delta f_\alpha}{\delta t} \right)_{coll} d^3 v = \left[\frac{\delta}{\delta t}(n_\alpha(\vv{r},t)<\chi>_\alpha)\right]
\end{equation}

Combinando as Eq. \ref{eq: pit5}, Eq. \ref{eq: pit7}, Eq. \ref{eq: pit10} e Eq. \ref{eq: pit11} na Eq. \ref{eq: pit3} produzimos então,
\begin{equation}
\label{eq: pit12}
\frac{\partial }{\partial t}(n_\alpha(\vv{r},t)<\chi>_\alpha) + \nabla \cdot (n_\alpha(\vv{r},t)<\chi \vv{v} >_\alpha) - n_\alpha(\vv{r},t)<\vv{a} \cdot \nabla_v \chi>_\alpha = 
\end{equation}
\begin{equation*}
=[\frac{\delta}{\delta t}(n_\alpha(\vv{r},t)<\chi>_\alpha)]
\end{equation*}

\chapter{Lista de Notações}
\label{Listanot}
Lista de notação:\\
$\vv{E}_i$ é o campo elétrico macroscópico interno\\
$\vv{B}_i$ é o campo magnético macroscópico interno\\
$\vv{B}(\vv{r})$ é o vetor campo magnético\\
$\vv{E}(\vv{r})$ é o campo elétrico\\
$\vv{J}(\vv{r},t)$ é vetor densidade de corrente\\
$\nu_{ionz}$ é a taxa de ionização\\
$\nu_{loss}$ é a taxa de perda de elétrons\\
$n_{\alpha} = \int{f_\alpha(\vv{r},\vv{v},t)}d\vv{v}$ densidade númerica da partícula do tipo $\alpha$\\
$\vv{u}_{\alpha} = \frac{1}{n(\vv{r},t)} \int_V{\vv{v} f_{\alpha}(\vv{r},\vv{v},t) d^3v}$ é a velocidade média das partículas em $v$\\
$n_i$ é a densidade númerica de íons\\
$n_e$ é a densidade númerica de elétrons\\
$m_\alpha$ massa do tipo de partícula $\alpha$\\
$\alpha_a$ é o 1º coeficiente de Townsend para o tipo de partícula $\alpha$ \\
$\tau_p$ é o tempo de confinamento da partícula\\
$f_\alpha(\vv{r},\vv{v},t)$ é a função distribuição de velocidades.\\
$\rho(\vv{r},t)$ é a densidade de carga\\
$Q_\alpha$ representa a taxa de mudança de densidade de energia devido ao espalhamento\\
$\mathcal{E}_\alpha$ é a produção de partículas de plasma\\
 $\mathbb{P}_\alpha = \rho_{m\alpha}<\vv{c}_\alpha \cdot \vv{c}_\alpha>$ é o tensor de pressão cinética\\ 
$\vv{q}_\alpha = \frac{1}{2} \rho_{m\alpha} <c^2_\alpha  \vv{c}_\alpha>$ é o fluxo de calor \\
$q_\alpha$ é carga da partícula do tipo $\alpha$ \\
$D_e$, $D_i$, $D$ sendo os coeficientes de difusão de partículas. \\
$p(\vv{r},t)$ é o escalar de pressão\\
$e$ é  carga elementar do eletron\\
$\eta$ sendo a resistividade paralela do Spitzer\\
$\vv{R}_\alpha$ termo de troca de momento\\
$R_\alpha$ denota a taxa de mudança de momento devido ao espalhamento\\
$\mathcal{R}_\alpha$ denota a taxa de mudança de momento devido à produção de partículas de plasma.
\\
Operadores diferenciais:
\begin{equation*}
\nabla = \hat{e}_x \frac{\partial}{\partial x} + \hat{e}_y \frac{\partial}{\partial y} + \hat{e}_z \frac{\partial}{\partial z}
\end{equation*}

\begin{equation*}
\nabla^2 = \hat{e}_x \frac{\partial^2}{\partial x^2} + \hat{e}_y \frac{\partial^2}{\partial y^2} + \hat{e}_z \frac{\partial^2}{\partial z^2}
\end{equation*}

\begin{equation*}
\nabla_v = \hat{e}_x \frac{\partial}{\partial v_x} + \hat{e}_y \frac{\partial}{\partial v_y} + \hat{e}_z \frac{\partial}{\partial v_z}
\end{equation*} 

\chapter{Valor Médio}
O símbolo $<$  $>_\alpha$ denota o valor médio com respeito ao espaço de velocidades para o tipo de párticula $\alpha$, o valor médio é independente de $\vv{v}$ mas é uma função de $\vv{r}$ e $t$.
\begin{equation}
<\chi(\vv{r},\vv{v},t)>_\alpha = \frac{1}{n_\alpha(\vv{r},t)} \int_V \chi(\vv{r},\vv{v},t) f_\alpha(\vv{r},\vv{v},t) d^3 v
\label{anexo1}
\end{equation}
onde $V$ representa o espaço de velocidade, ou seja, todas as velocidades possíveis para cada partícula, $n_\alpha(\vv{r},t)=\int_V{f_\alpha(\vv{r},\vv{v},t)}d\vv{v}$ e a densidade númerica de partículas do tipo $\alpha$ na posição $\vv{r}$ no tempos $t$ e $f_\alpha(\vv{r},\vv{v},t)$ é a função distribuição de velocidades para as partículas do tipo $\alpha$.

\end{document}
